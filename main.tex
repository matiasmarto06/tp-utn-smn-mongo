\documentclass[11pt, a4paper]{article}

% --- PAQUETES REQUERIDOS ---
\usepackage[utf8]{inputenc}
\usepackage[spanish, es-tabla]{babel}
\usepackage[T1]{fontenc}
\usepackage{geometry}
\geometry{top=2.5cm, bottom=2.5cm, left=2.5cm, right=2.5cm}
\usepackage{lmodern}
\usepackage{hyperref}
\usepackage{listings}
\usepackage{xcolor}
\usepackage{parskip}
\usepackage{graphicx}

% --- CONFIGURACIÓN DE COLORES ---
\definecolor{codegreen}{rgb}{0,0.6,0}
\definecolor{codegray}{rgb}{0.5,0.5,0.5}
\definecolor{codepurple}{rgb}{0.58,0,0.82}
\definecolor{backcolour}{rgb}{0.95,0.95,0.92}
\definecolor{delim}{RGB}{20,105,176}
\definecolor{numb}{RGB}{106, 109, 32}

% --- DEFINICIÓN DE LENGUAJES ---
\lstdefinelanguage{json}{
    basicstyle=\normalfont\ttfamily\footnotesize,
    numbers=left,
    numberstyle=\tiny,
    stepnumber=1,
    numbersep=8pt,
    showstringspaces=false,
    breaklines=true,
    frame=lines,
    backgroundcolor=\color{backcolour},
    literate=
     *{0}{{{\color{numb}0}}}{1}
      {1}{{{\color{numb}1}}}{1}
      {2}{{{\color{numb}2}}}{1}
      {3}{{{\color{numb}3}}}{1}
      {4}{{{\color{numb}4}}}{1}
      {5}{{{\color{numb}5}}}{1}
      {6}{{{\color{numb}6}}}{1}
      {7}{{{\color{numb}7}}}{1}
      {8}{{{\color{numb}8}}}{1}
      {9}{{{\color{numb}9}}}{1}
      {:}{{{\color{delim}{:}}}}{1}
      {,}{{{\color{delim}{,}}}}{1}
      {\{}{{{\color{delim}{\{}}}}{1}
      {\}}{{{\color{delim}{\}}}}}{1}
      {[}{{{\color{delim}{[}}}}{1}
      {]}{{{\color{delim}{]}}}}{1},
}

\lstdefinelanguage{javascript}{
    keywords={typeof, new, true, false, catch, function, return, null, catch, switch, var, if, in, while, do, else, case, break, db, find, findOne, insertMany, insertOne, deleteOne, replaceOne, updateOne, countDocuments, set},
    keywordstyle=\color{blue}\bfseries,
    ndkeywords={class, export, boolean, throw, implements, import, this, ObjectId},
    ndkeywordstyle=\color{darkgray}\bfseries,
    identifierstyle=\color{black},
    sensitive=false,
    comment=[l]{//},
    morecomment=[s]{/*}{*/},
    commentstyle=\color{codegreen}\ttfamily,
    stringstyle=\color{codepurple}\ttfamily,
    morestring=[b]',
    morestring=[b]"
}

\lstdefinestyle{mystyle}{
    backgroundcolor=\color{backcolour},
    commentstyle=\color{codegreen},
    keywordstyle=\color{magenta},
    numberstyle=\tiny\color{codegray},
    stringstyle=\color{codepurple},
    basicstyle=\ttfamily\footnotesize,
    breakatwhitespace=false,
    breaklines=true,
    captionpos=b,
    keepspaces=true,
    numbers=left,
    numbersep=5pt,
    showspaces=false,
    showstringspaces=false,
    showtabs=false,
    tabsize=2
}
\lstset{style=mystyle}

\begin{document}

\begin{titlepage}
    \centering

    \vspace*{1cm}
    \begin{figure}[h]
        \centering
        \includegraphics[width=8cm]{logo.png} 
    \end{figure}
    \vspace*{1cm}
    
    \vspace{1cm}
    {\scshape\Large Tecnicatura Universitaria en Programación \par}
    \vspace{1.5cm}
    
    {\huge\bfseries Informe Técnico: Operaciones CRUD en Colección Stations \par}
    \vspace{0.5cm}
    {\Large Trabajo Práctico de Base de Datos NoSQL \par}
    \vspace{2cm}
    
    \begin{tabular}{rl}
        \Large \textbf{Materia:} & \Large Base de Datos II \\ [0.5cm]
        \Large \textbf{Profesor:} & \Large Ángel Simón \\ [0.5cm]
        % \Large \textbf{JTP:} & \Large [Nombre del JTP si aplica] \\
    \end{tabular}
    
    \vfill
    
    {\Large \textbf{Alumno:} Matías Ignacio Martorano \par}
    \vspace{0.5cm}
    {\Large \textbf{Legajo:} 27526 \par}
    \vspace{1.5cm}
    
    {\Large Año 2025 \par}

\end{titlepage}

\section{Colección: stations}

Almacena la información estática de las estaciones meteorológicas.

\subsection{Estructura del Documento}

\begin{lstlisting}[language=json, caption=Ejemplo de documento completo en stations]
{
  "_id": ObjectId("672f3a4d1b5e8a4a2c98d09f"),
  "Code": "89055",
  "OACI": "SAWB",
  "Name": "BASE MARAMBIO",
  "Province": "ANTARTIDA",
  "Latitude": -64.2333,
  "Longitude": -56.6167,
  "Altitude": 198
}
\end{lstlisting}

\section{Operaciones Implementadas}

\subsection{Insertar Documento Único (`InsertOneAsync`)}

Utilizado para agregar una nueva estación manualmente.

\begin{itemize}
    \item \textbf{C\# (Driver):}
    \begin{lstlisting}[language={[Sharp]C}]
await Stations.InsertOneAsync(station);
    \end{lstlisting}
    \item \textbf{Mongo Shell:}
    \begin{lstlisting}[language=javascript]
db.stations.insertOne({
    "Code": "88963",
    "OACI": "SAYE",
    "Name": "BASE ESPERANZA",
    "Province": "ANTARTIDA",
    "Latitude": -63.3978,
    "Longitude": -56.9972,
    "Altitude": 24
})
    \end{lstlisting}
\end{itemize}
\newpage
\subsection{Insertar Múltiples Documentos (`InsertManyAsync`)}

Utilizado para la carga inicial masiva de estaciones.

\begin{itemize}
    \item \textbf{C\# (Driver):}
    \begin{lstlisting}[language={[Sharp]C}]
await Stations.InsertManyAsync(stations);
    \end{lstlisting}
    \item \textbf{Mongo Shell:}
    \begin{lstlisting}[language=javascript]
db.stations.insertMany([
    {
        "Code": "89034",
        "OACI": "SAYB",
        "Name": "BASE BELGRANO II",
        "Province": "ANTARTIDA",
        "Latitude": -77.8739,
        "Longitude": -34.6275,
        "Altitude": 256
    },
    {
        "Code": "89066",
        "OACI": "SAYS",
        "Name": "BASE SAN MARTIN",
        "Province": "ANTARTIDA",
        "Latitude": -68.1300,
        "Longitude": -67.1000,
        "Altitude": 7
    }
])
    \end{lstlisting}
\end{itemize}

\subsection{Leer Todos los Documentos (`Find` sin filtros)}

Utilizado para listar todas las estaciones disponibles.

\begin{itemize}
    \item \textbf{C\# (Driver):}
    \begin{lstlisting}[language={[Sharp]C}]
await Stations.Find(_ => true).ToListAsync();
    \end{lstlisting}
    \item \textbf{Mongo Shell:}
    \begin{lstlisting}[language=javascript]
db.stations.find({})
    \end{lstlisting}
\end{itemize}
\subsection{Leer un Documento Específico (`Find` con filtro)}

Utilizado para obtener los detalles de una estación por su ID.

\begin{itemize}
    \item \textbf{C\# (Driver):}
    \begin{lstlisting}[language={[Sharp]C}]
var filter = Builders<Station>.Filter.Eq(s => s.Id, id);
await Stations.Find(filter).FirstOrDefaultAsync();
    \end{lstlisting}
    \item \textbf{Mongo Shell:}
    \begin{lstlisting}[language=javascript]
db.stations.findOne({ "_id": ObjectId("672f3a4d1b5e8a4a2c98d09f") })
    \end{lstlisting}
\end{itemize}

\subsection{Actualizar un Documento (`ReplaceOneAsync`)}

Reemplaza el documento completo con los nuevos valores editados.

\begin{itemize}
    \item \textbf{C\# (Driver):}
    \begin{lstlisting}[language={[Sharp]C}]
var filter = Builders<Station>.Filter.Eq(s => s.Id, station.Id);
await Stations.ReplaceOneAsync(filter, station);
    \end{lstlisting}
    \item \textbf{Mongo Shell:}
    \begin{lstlisting}[language=javascript]
db.stations.replaceOne(
    { "_id": ObjectId("672f3a4d1b5e8a4a2c98d09f") },
    {
        "Code": "89055",
        "OACI": "SAWB",
        "Name": "BASE MARAMBIO MODIFICADA",
        "Province": "ANTARTIDA",
        "Latitude": -64.2333,
        "Longitude": -56.6167,
        "Altitude": 200
    }
)
    \end{lstlisting}
\end{itemize}

\subsection{Eliminar un Documento (`DeleteOneAsync`)}

Elimina una estación específica.

\begin{itemize}
    \item \textbf{C\# (Driver):}
    \begin{lstlisting}[language={[Sharp]C}]
var filter = Builders<Station>.Filter.Eq(s => s.Id, id);
await Stations.DeleteOneAsync(filter);
    \end{lstlisting}
    \item \textbf{Mongo Shell:}
    \begin{lstlisting}[language=javascript]
db.stations.deleteOne({ "_id": ObjectId("672f3a4d1b5e8a4a2c98d09f") })
    \end{lstlisting}
\end{itemize}

\subsection{Contar Documentos (`CountDocumentsAsync`)}

Muestra el total de estaciones.

\begin{itemize}
    \item \textbf{C\# (Driver):}
    \begin{lstlisting}[language={[Sharp]C}]
await Stations.CountDocumentsAsync(_ => true);
    \end{lstlisting}
    \item \textbf{Mongo Shell:}
    \begin{lstlisting}[language=javascript]
db.stations.countDocuments({})
    \end{lstlisting}
\end{itemize}

\newpage

\section{Colección: measurements}

Esta colección almacena los datos meteorológicos horarios. Para optimizar el almacenamiento y permitir la normalización, cada documento de medición guarda una \textbf{referencia} a la estación meteorológica correspondiente, en lugar de duplicar toda la información de la estación.

\subsection{Estructura del Documento}

\begin{lstlisting}[language=json, caption=Ejemplo de documento en measurements]
{
  "_id": ObjectId("672f3e8b1b5e8a4a2c98d0a1"),
  "station_id": ObjectId("672f3a4d1b5e8a4a2c98d09f"),
  "time": ISODate("2025-10-27T14:00:00.000Z"),
  "temp": -3.5,
  "dwpt": -4.9,
  "rhum": 90,
  "prcp": 0.0,
  "snow": null,
  "wdir": 25,
  "wspd": 14.8,
  "wpgt": null,
  "pres": 993.5,
  "tsun": null,
  "coco": 5
}
\end{lstlisting}

\section{Operaciones Implementadas}

\subsection{Inserción Masiva de Datos (`insertMany`)}

El colector de datos obtiene 24 mediciones de una sola vez por cada estación. Para eficiencia, estas se insertan en una sola operación por lotes.

\begin{itemize}
    \item \textbf{C\# (Driver):}
    \begin{lstlisting}[language={[Sharp]C}]
await Measurements.InsertManyAsync(measurements, new InsertManyOptions { IsOrdered = false });
    \end{lstlisting}
    \item \textbf{Mongo Shell:}
    \begin{lstlisting}[language=javascript]
db.measurements.insertMany([
   {
     "station_id": ObjectId("672f3a4d1b5e8a4a2c98d09f"),
     "time": ISODate("2025-11-08T10:00:00Z"),
     "temp": -5.4,
     "dwpt": -6.1,
     "rhum": 92,
     "prcp": 0.0,
     "snow": null,
     "wdir": 220,
     "wspd": 20.5,
     "wpgt": 35.0,
     "pres": 988.5,
     "tsun": null,
     "coco": 7
   },
   {
     "station_id": ObjectId("672f3a4d1b5e8a4a2c98d09f"),
     "time": ISODate("2025-11-08T11:00:00Z"),
     "temp": -5.1,
     "dwpt": -6.0,
     "rhum": 90,
     "prcp": 0.0,
     "snow": null,
     "wdir": 215,
     "wspd": 18.2,
     "wpgt": 32.0,
     "pres": 989.0,
     "tsun": null,
     "coco": 7
   }
], { ordered: false })
    \end{lstlisting}
\end{itemize}

\subsection{Inserción de Documento Único (`insertOne`)}

Aunque menos común, el sistema permite insertar una medición individual si fuera necesario.

\begin{itemize}
    \item \textbf{C\# (Driver):}
    \begin{lstlisting}[language={[Sharp]C}]
await Measurements.InsertOneAsync(measurement);
    \end{lstlisting}
    \item \textbf{Mongo Shell:}
    \begin{lstlisting}[language=javascript]
db.measurements.insertOne({
    "station_id": ObjectId("672f3a4d1b5e8a4a2c98d09f"),
    "time": new Date(),
    "temp": -4.0,
    "dwpt": -5.5,
    "rhum": 85,
    "prcp": 0.0,
    "snow": null,
    "wdir": 210,
    "wspd": 15.0,
    "wpgt": 28.0,
    "pres": 990.1,
    "tsun": null,
    "coco": 3
})
    \end{lstlisting}
\end{itemize}
\newpage
\subsection{Lectura de Todos los Documentos (`find` sin filtros)}

Utilizada para propósitos de depuración o volcado completo de datos.

\begin{itemize}
    \item \textbf{C\# (Driver):}
    \begin{lstlisting}[language={[Sharp]C}]
await Measurements.Find(_ => true).ToListAsync();
    \end{lstlisting}
    \item \textbf{Mongo Shell:}
    \begin{lstlisting}[language=javascript]
db.measurements.find({})
    \end{lstlisting}
\end{itemize}

\subsection{Consulta con Filtros Múltiples (`find` / `$match$`)}

Utilizada por el módulo de gráficos para obtener los datos crudos antes de procesarlos. Permite filtrar por estación, rango de fechas y valores específicos de variables.

\begin{itemize}
    \item \textbf{C\# (Driver):}
    \begin{lstlisting}[language={[Sharp]C}]
var filter = Builders<Measurement>.Filter.Eq(m => m.StationId, ObjectId.Parse(stationId)) &
             Builders<Measurement>.Filter.Gte(m => m.Time, dateFrom) &
             Builders<Measurement>.Filter.Lte(m => m.Time, dateTo);

if (variableName != null && criteria != null && value.HasValue) {
    // Se agrega dinamicamente el filtro por variable
    // Ejemplo: filter &= builder.Gte("temp", -5.0);
}

await Measurements.Find(filter).ToListAsync();
    \end{lstlisting}
    \item \textbf{Mongo Shell:}
    \begin{lstlisting}[language=javascript]
db.measurements.find({
    "station_id": ObjectId("672f3a4d1b5e8a4a2c98d09f"),
    "time": {
        "$gte": ISODate("2025-11-01T00:00:00Z"),
        "$lte": ISODate("2025-11-08T23:59:59Z")
    },
    "temp": { "$gt": -5.0 }
})
    \end{lstlisting}
\end{itemize}
\newpage
\subsection{Unión con Estaciones, Filtrado y Ordenamiento (Aggregation Pipeline)}

Esta es la operación más compleja y completa del sistema, utilizada por la grilla principal de datos. Combina múltiples etapas del pipeline de agregación para obtener una vista enriquecida de los datos.

\begin{itemize}
    \item \textbf{C\# (Driver):}
    \begin{lstlisting}[language={[Sharp]C}]
var pipeline = Measurements.Aggregate()
    .Match(filter)          //$match (filtrado inicial)
    .Sort(sortDefinition)   //$sort (ordenamiento dinamico)
    .Lookup(                //$lookup (union con stations)
        foreignCollectionName: "stations",
        localField: "station_id",
        foreignField: "_id",
        as: "station_joined"
    )
    .Unwind("station_joined") // $unwind (aplanar array)
    .As<BsonDocument>();      // Proyeccion final
    \end{lstlisting}
    \item \textbf{Mongo Shell:}
    \begin{lstlisting}[language=javascript]
db.measurements.aggregate([
    {
        // Filtrar por estacion y fecha
        "$match": {
            "station_id": ObjectId("672f3a4d1b5e8a4a2c98d09f"),
            "time": {
                "$gte": ISODate("2025-11-01T00:00:00Z"),
                "$lte": ISODate("2025-11-08T23:59:59Z")
             }
        }
    },
    {
        // Ordenar por fecha descendente (mas reciente primero)
        "$sort": { "time": -1 }
    },
    {
        // Unir con la coleccion 'stations' para obtener el nombre
        "$lookup": {
            "from": "stations",
            "localField": "station_id",
            "foreignField": "_id",
            "as": "station_joined"
        }
    },
    {
        // 'Aplanar' el array resultante de la union para facilitar su uso
        "$unwind": "$station_joined"
    }
])
    \end{lstlisting}
\end{itemize}

\newpage

\section{Colección: api\_requests}

Almacena un registro detallado de cada llamada realizada a la API externa de Meteostat. Esta colección es fundamental para la auditoría y el control de la cuota de uso mensual.

\subsection{Estructura del Documento}

\begin{lstlisting}[language=json, caption=Ejemplo de documento en api\_requests]
{
  "_id": ObjectId("672f3b12a1b2c3d4e5f67890"),
  "station": {
    "_id": ObjectId("672f3a4d1b5e8a4a2c98d09f"),
    "Code": "89055",
    "OACI": "SAWB",
    "Name": "BASE MARAMBIO",
    "Province": "ANTARTIDA",
    "Latitude": -64.2333,
    "Longitude": -56.6167,
    "Altitude": 198
  },
  "timestamp": ISODate("2025-11-08T14:30:00Z"),
  "isScheduled": true,
  "success": true,
  "message": "OK"
}
\end{lstlisting}

\section{Operaciones Implementadas}

\subsection{Insertar Registro (`InsertOneAsync`)}

Utilizado por el colector automático para registrar cada intento de conexión con la API, independientemente de si tuvo éxito o falló.

\begin{itemize}
    \item \textbf{C\# (Driver):}
    \begin{lstlisting}[language={[Sharp]C}]
await Logs.InsertOneAsync(log);
    \end{lstlisting}
    \item \textbf{Mongo Shell:}
    \begin{lstlisting}[language=javascript]
db.api_requests.insertOne({
    "station": {
        "_id": ObjectId("672f3a4d1b5e8a4a2c98d09f"),
        "Code": "89055",
        "Name": "BASE MARAMBIO",
        // ... Resto de los campos ...
    },
    "timestamp": new Date(),
    "isScheduled": true,
    "success": true,
    "message": "OK"
})
    \end{lstlisting}
\end{itemize}
\newpage
\subsection{Leer Registros por Mes (`Find` con filtro de fecha)}

Utilizado para consultar el historial de llamadas realizadas en un período específico (por ejemplo, un mes calendario).

\begin{itemize}
    \item \textbf{C\# (Driver):}
    \begin{lstlisting}[language={[Sharp]C}]
var start = new DateTime(year, month, 1);
var end = start.AddMonths(1);
var filter = Builders<ApiRequestLog>.Filter.Gte(l => l.Timestamp, start) &
             Builders<ApiRequestLog>.Filter.Lt(l => l.Timestamp, end);

await Logs.Find(filter).ToListAsync();
    \end{lstlisting}
    \item \textbf{Mongo Shell:}
    \begin{lstlisting}[language=javascript]
db.api_requests.find({
    "timestamp": {
        "$gte": ISODate("2025-11-01T00:00:00Z"),
        "$lt": ISODate("2025-12-01T00:00:00Z")
    }
})
    \end{lstlisting}
\end{itemize}

\subsection{Contar Llamadas Mensuales (`CountDocumentsAsync`)}

Operación crítica utilizada antes de cada ciclo de sondeo para verificar si se ha alcanzado el límite mensual de 500 llamadas a la API gratuita.

\begin{itemize}
    \item \textbf{C\# (Driver):}
    \begin{lstlisting}[language={[Sharp]C}]
var start = new DateTime(DateTime.Now.Year, DateTime.Now.Month, 1);
var end = start.AddMonths(1);
var filter = Builders<ApiRequestLog>.Filter.Gte(l => l.Timestamp, start) &
             Builders<ApiRequestLog>.Filter.Lt(l => l.Timestamp, end);

await Logs.CountDocumentsAsync(filter);
    \end{lstlisting}
    \item \textbf{Mongo Shell:}
    \begin{lstlisting}[language=javascript]
db.api_requests.countDocuments({
    "timestamp": {
        "$gte": ISODate("2025-11-01T00:00:00Z"),
        "$lt": ISODate("2025-12-01T00:00:00Z")
    }
})
    \end{lstlisting}
\end{itemize}

\end{document}